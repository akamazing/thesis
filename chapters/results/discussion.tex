\section{Summary of Contributions}
In this paper we identified five metamorphic properties "Rotation", "Shading", "Shearing", "Shifting the image along $X$-axis", and, "Shifting the image along $Y$-axis" that can be used to transform image datasets to create follow-up test cases.  We applied these transformations on three datasets and calculated their accuracy on five common machine learning algorithms. We found that the individual classes are affected differently depending on the transformation. Hence, we provided recommendations to select the parameters of our properties such that the follow-up test cases are similar to the original dataset. 
We also evaluated the robustness of the machine learning algorithms on the follow-up test data. From this evaluation, we found convolutional neural-networks to be more robust on the MNIST dataset than the other machine learning algorithms we implemented. However, this trend does not carry over to the other datasets. For both the Fashion-MNIST and EMNIST-Letter datasets, the Naive Bayes algorithm showed more robustness. We also found that the first three properties: rotate, shade, and shear, are not as sensitive as the shiftX and shiftY properties.
% These functions can be used as frameworks to . 
% Future researchers can utilize these functions to plugin more algorithms and metamorphic properties.
During the study, we faced the challenge of defining the concept of "similarity" of test cases. There is no universally agreed-upon definition of similarity, and its definition depends upon the context of use. For this study, the follow-up datasets that produced accuracy within $90\%$ of the original dataset are considered similar to the original dataset. Future researchers can change the definition of "similarity" to produce different results.

% We also investigated the robustness of the five machine learning algorithms on metamorphic transformations.

This research also provides a methodological contribution where we have generated a framework, which can be easily extended to add more metamorphic properties, datasets, and algorithms for future research. All the functions, datasets, and algorithms used in this project are hosted on the project's Github page \cite{abhishek2020}. Further researches can be done to combine these transformations to create even more complex transformations by using the output of one transformation as input to others.
% Contributions:
% Methodological contribution where others can plugin more algors or MT.
% Experimental contribution:Which algo does better.

\section{Conclusion}
In this study, we set to investigate three main questions: can we identify new metamorphic properties to be used with image datasets? Do these properties affect all the classes equally? And what range of transformation can be applied, such that the follow-up test data remains similar to original data? We answered the first question by naming five affine transformations, which can be used as metamorphic properties. We also implemented them on three datasets to generate new test cases. To answer the second research question, we implemented five machine learning algorithms and tested the accuracy of the new test cases for each class. We found that the classes had different areas under curve for metamorphic transformations; thus, they are affected differently by the different metamorphic transformations. 
We also made recommendations to select parameters for our metamorphic transformations such that the resulting follow-up test cases are similar to the original dataset. These recommendations can be used as a test case generation strategy to generate additional transformed test data for MNIST, Fashion-MNIST, and EMNIST-Letter dataset. Future researches should consider using our metamorphic transformations and recommendations to generate new test cases for their machine learning algorithms.