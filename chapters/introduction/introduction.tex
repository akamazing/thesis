Machine learning has gained rapid popularity in the past decade and different sectors like healthcare, finance, retail, etc. are using machine learning to provide better products and services. This rising popularity has created a demand for better implementation of the state-of-the-art algorithms in order to implement more complex and sophisticated use cases. Machine learning has been around for a long time now and several developers have written a number of tools and libraries to help others learn and use these algorithms in their own projects.  While developing and training a machine learning model, the developers rely on the accuracy of the libraries to produce best results. The aim is to create a model that makes the best predictions.
Conventionally, oracles have been used to test the correctness of a program. Oracles provide expected values against which the output from the program can be compared and validated. Lack of reliable oracles for testing machine learning algorithms makes it very hard to test the accuracy of such programs. This problem is called the oracle problem\cite{Weyuker}. Oracle problem arises when either:
\begin{itemize}
  \item An oracle does not exist, or,
  \item An oracle can theoretically exist but it is computationally too expensive to determine the output.
\end{itemize}
Such set of programs which does not have a test oracle to predict the output on a set of inputs are called \enquote{non-testable programs}\cite{Murphy}. Davis and Weyuker describe these set of programs as \enquote{Programs which were written in order to determine the answer in the first place. There would be no need to write such programs if the correct answer were known}\cite{Davis1981}. Most machine learning programs fall under this category as they are written in order to predict the correct answers in the first place. Several techniques can be employed to verify the correctness of such programs. One of the most popular methods being the use of pseudo-oracles. To use pseudo-oracle two or more implementations of the algorithm are independently developed to fulfill the same specification. They are run on the same input test data and the outputs from them are then compared. If the outputs match it can be asserted that the original results are according to the specification. This kind of testing is often referred to as dual coding. However, this technique introduces a significant overhead in terms of implementing two or more versions of the same algorithm and testing their outputs and is more suitable for mission-critical systems\cite{Weyuker}.\newline

Another testing methodology called Metamorphic testing introduced by Chen et. al can be used for testing ML programs instead. It addresses some of these problems while testing the \enquote{non-testable programs} without significant overhead. The idea behind Metamorphic testing is that it easier to compare and understand the relationship between outputs than to compare and understand input-output behavior. For a prototype example: To test a program which implements the $sin$ function, the output from the program for value of $sin$ at $x$ could be compare to the real value of $sin(x)$ or, the mathematical property of $sin(x) = sin(\pi-x)$ can be exploited to verify the correct implementation of $sin$ function. Such relation $(sin(x) = sin(\pi-x))$ are called Metamorphic Relations. Metamorphic testing makes use of metamorphic relations where an input relation is used to generate new input test cases from existing test data, and an output relation is used to compare the outputs produced by the test cases.\newline
In this paper, we will explore the types of guarantees one can expect a machine learning model to possess because of the properties that the underlying algorithm of the implementation.
% In particular we will look at google's tensorflow.
% Dataset coverages
% and
% We propose applying MT as a way of testing and verifying ML programs.

%ML programs importance and them following under non testable. Need for Testing ML programs.  Why current techniques are not enough and introduce MT. Use MT for testing some popular implementation of ML.

In this study we will answer the following research questions:
\begin{enumerate}
%   \item How does metamorphic testing compare to other testing methodologies for testing machine learning algorithms?
  %\item Can we quantify reliability of predictions made by machine learning algorithms?
  \item Identifying metamorphic properties that can be used to generate new test data from image datasets.
%   \item How does metamorphic properties affect different dataset and deep-learning algorithms?
  \item Robustness of the deep learning algorithms to the identified metamorphic properties?
  \item What range of Metamorphic properties identified is most useful while generating new test cases.
%   \item How sensitive are the predictions made by ML algorithms to the change in input data?
%   \item At what point does the algorithm fail to produce correct classifications on distorted inputs?
  
\end{enumerate}
%Point of failure. by humans& computer. (visual inpection)
%Properties of ML affect
%
